\label{chapter 3}
A multi-agent simulation aims to integrate dynamic traffic assignment with the demand generated by the daily activities of individuals, resulting in a need for transport \citep{erath2012large}. Within the contexts of freight modelling various agents are involved. This includes, but is not limited to, shippers, administrators carriers and receivers. Understanding this multi-agent interaction is made possible by \acrshort{matsim}. However, the inclusion of freight vehicles in \acrshort{matsim} is not as developed as that of passenger agents. Many researchers have contributed to this field of research by extending the \acrshort{matsim} framework. It is therefore necessary to investigate the infrastructure currently available in \acrshort{matsim}. 

 %In order to achieve this, the \acrshort{matsim} platform will be investigated. This includes the existing software layers, the carrier and receiver agent, as well as the inclusion of behavioural modelling in urban freight transportation models. From here, the receiver-carrier collaboration in \acrshort{matsim} will be investigated to gain an understanding of the multi-agent implementation and how this is incorporated in receiver-carrier collaborations and the accompanied cost allocation of resulting benefits. This will initiate an investigation on the current status of cost allocation methods in \acrshort{matsim} and a discussion on the proposed method to be embedded. In order to test the functionality of the model, a small scenario and testing methodology is discussed in the simulation setup.

%---------------------------------------------------------------------------------------------------------------------------------------------------------------------------------------------------------------

\section{The MATSim framework}
\label{matsim_framework}
\acrshort{matsim} is an open-source multi-agent simulation framework. The framework is implemented in Java and has the ability to run large scale models. By default, the model is structured to run a single day, 24 hour, simulation but as \acrshort{matsim} allows for extensions, a multi-day simulation is possible. \acrshort{matsim} is an activity based simulation framework that is based on the co-evolutionary principle. This co-evolutionary platform allows agents to optimise their own schedules, throughout the simulation, while competing with other agents. With each repeated iteration, an agent continuously optimizes its daily activities and schedules to improve its own situation. A simulation run consists of a customisable number of iterations that are executed in a loop, this is referred to as the \acrshort{matsim} cycle as shown in figure X.

FIGURE OF MATSIM CYCLE

The \acrshort{matsim} cycle consists of five building blocks. The simulation process is started with the \textit{initial demand}. The daily activities of the study area population, made up of agents, are derived from activity chains. These activity chains are obtained by means of data sampling or discrete event modelling. With each iteration, each individual agent improves this initial demand \citep{horni2016multi}. \citet{horni2016multi} continues by identifying the second building block in the \acrshort{matsim} cycle \textit{mobsim}, that refers to the mobility simulation. During this stage, each agent in the population selects a plan from its memory, that contains a customisable number of daily plans. The agent executes the selected plan, composed of the daily activity chain. The performance of the selected plan is then scored in the next stage, \textit{scoring}. The plan selection, in the mobsim stage, depends on the scores allocated to these plans. The performance of the plan is scored as an econometric utility. After a score has been assigned to each plan executed in the mobsim, each agent selects a new plan from its memory, in the \textit{replanning} stage. Once again, this selection depends on the scores allocated to the list of plans stored in the agents' memory. A customisable share of agents (usually 10\%) have the ability to not only select a new plan, but customise a plan, in order to improve its performance.  This selection of agents clone the selected plan and modifies certain parameters of this clone. On completion, the new plan is also stored in the agents' memory. Parameters considered during the plan modification include route selection, mode choice, time choice and destination choice (refer to \citet{horni2016multi} for further research on the different choice dimensions currently under development.)\par

In the instance where an agents' memory exceeds the defined number of plans to be stored, the plan with the lowest allocated score is removed from it's memory. Once the agents have selected a new plan from existing plans or modified a plan during the replanning stage, these plans are once again executed in the mobsim, initiating a new iteration. This process is executed until the specified number of iterations is completed and the \acrshort{matsim} cycle enters the last building block, \textit{analyses}. during this final stage the performance of the different plans are evaluated.\par

  
%---------------------------------------------------------------------------------------------------------------------------------------------------------------------------------------------------------------

\section{Logistic Behavioural Modelling in MATSim}
 The \acrshort{matsim} framework allows external parties to customize and extend the functionality in \acrshort{matsim}. The modular architecture of this software allows modellers to contribute to various levels of the framework. In chapter \ref{chapter2}, various contributions to the field of urban freight modelling in \acrshort{matsim} was discussed.
  Of these listed, the contributions made by CARRIER and RECEIVER, paved the way for new research areas concerning behavioral modelling in \acrshort{matsim}. Prior to these contributions, commercial vehicles were simulated as background load on the road network. Furthermore, simulating commercial vehicles were limited as the movement of these vehicles could not be altered by the plan modifications and replanning stage available to private vehicles. The carrier agent and the receiver agent, provided by CARRIER and RECEIVER, respectively,  play a distinctive role in modelling behaviour of urban freight vehicles.

%-------------------------------------------------------------------------------------------------------

\subsection{The carrier agent}
\citet{schroeder2012towards} introduced a new software layer, the \textit{carrier} agent. This agent is responsible for the transportation of goods from a certain origin to a destination location. The carrier receives certain pick-up and delivery time windows during which he must complete his services at each of the receiver locations. Each carrier agent has a set of vehicles allocated to him. Each of these vehicles are tasked with a schedule that describes the list of receivers and the specified pick-up, delivery and arrival times, as planned by the carrier. The schedule also contains the planned route through the network.\par

Consider the \acrshort{matsim} cycle in figure XX. To incorporate the carrier agent into this cycle, a toolkit called \texttt{jsprit} is used in conjunction with \acrshort{matsim}, as shown in figure XX. \texttt{jsprit} is an open source Java toolkit used to solve Vehicle Routing and Travelling Salesman problems JSPRIT GITHUB REFERENCE.  This figure was adapted from BEAN. The \acrshort{matsim} and \texttt{jsprit} interface cycle also consists of five building blocks. Prior to executing the \textit{mobsim} stage, \texttt{jsprit} is called to construct the schedules for each carrier.
 \par

ADD MATSIM CYCLE WITH CARRIER EXTENSION

%For a complete review of the carrier architecture please refer to BEAN and HER SOURCE.

The carriers data container creates the interface with \texttt{jsprit}. Each of these carriers are assigned certain capabilities when executing activities, including the different vehicle types and fleet size available to that carrier. Each carrier operates from allocated depots, where each depot is assigned with a fleet size and vehicle type. This allocates certain vehicles to a certain depot location along with the specified depot operation times. Therefore, the carrier makes use of a variety of vehicles to perform pickup and delivery activities. If the carrier agent acts as both the carrier and the shipper, the depot location will act as the pickup location (shipper location). The carrier will only perform delivery activities in such an instance. All pickup and delivery activities are assigned a certain pickup and delivery time windows.\par
Once the carrier capabilities and pickup and/or delivery time windows are assigned to each of the carriers, it is necessary to determine the delivery schedule, referred to as the tour schedule. The tour schedule contains the allocated vehicle, the planned departure times and the pickup and delivery sequence for each carrier. The carrier can combine different orders into a single vehicle, servicing all the relevant customers in a single delivery trip. However, a single order cannot be split into multiple freight vehicles. The pick-up and delivery sequence for multiple orders is determined by solving the Vehicle Routing and Travelling Salesman problem in \texttt{jsprit}. The resulting tour schedules are routed on the network to construct the routed tour schedules (referred to as the carrier plan) used by carrier agents to execute delivery activities in a single day. The completed carrier plan serves as the second interface between \texttt{jsprit} and \textit{matsim}. The carrier plan is injected into the mobsim. The agents travelling on the road network and executing assigned plans can be seen as individual drivers or rather a combination of a vehicle and a driver. One carrier agent can therefore have multiple vehicles travelling on the network.\par

Based on the carrier plan, each vehicle leaves its allocated depot at the planned time and travels on the network to the first customer stipulated on the list. The vehicle travels the shortest path on the network in order to save time and reduce the delivery cost. The shortest path is determined by the \texttt{jsprit} toolkit. Each schedule contains the planned arrival time (delivery time). This is, however, not necessarily the actual arrival time at the location. Thus, the vehicle could arrive earlier or later than the planned arrival time. Each receiver is allocated a delivery time window in which deliveries are accepted. If the vehicle arrives prior to the start time of the delivery time window, the vehicle has to wait at the location until the prescribed delivery window start time. If the vehicle exceeds the planned travel time and therefore missed the delivery time window by being too late, the vehicle receives a penalty cost. If the vehicle arrives in the designated delivery time window, the delivery starts, initiating the unloading time. If multiple deliveries are consolidated into a single vehicle, the vehicle will complete the deliveries, in the sequence stipulated in the tour schedule, by travelling on the network to the different receiver locations. If all deliveries on the tour schedule for the specific vehicle has been completed, the vehicle returns to its assigned depot location. \par

What makes agent-based modelling such a powerful tool is the means in which different agents interact, resulting in underlying behaviour being surfaced. As one carrier could have multiple vehicles travelling on the network, the individual vehicles have to compete with each other for road space. The framework also allows for multiple carriers to be included in a single simulation, resulting in a single vehicle competing with not only its own carriers' vehicles but also other vehicles belonging to additional carriers injected into the simulation. As these commercial vehicles, or rather freight agents, compete with one another, another group of agents could be injected into the simulation to represent the private vehicles on the network. The inclusion of different agents competing on the network, results in elongated travel times as congestion arises from the interacting agents. \par

After all plans have been executed by the different vehicles, the performance of the plans are evaluated and scored as an econometric utility function. After the \textit{scoring} stage has been completed, the carrier agent attempts to improve the performance of vehicles by altering the plans in the \textit{replanning} stage. The carrier will change the scheduling of the vehicles and switch the allocated deliveries between different vehicles in order to reduce the accumulated delivery cost. This process is repeated until the specified number of iterations is completed and the final stage, \textit{analyses} evaluates the performance of the different plans.

%-------------------------------------------------------------------------------------------------------

\subsection{The receiver agent}
Until recently, the receivers in \acrshort{matsim} were included statistically. Receiver requirements were added as a list in the services to be completed by the carrier agent. The service included the receiver's order quantity, location, service time and specified delivery time window. The carrier uses this information to construct the tour schedule. Therefore, the entire simulation and delivery schedule is based on the receiver requirements. Changing any of the receiver's parameters will have a significant affect on the services required and the resulting tour schedule. BEAN addressed this area for improvement by identifying the need for additional behavioural elements of receivers in \acrshort{matsim}. This will enable modellers to include realistic interactions between the carrier and receiver agent. BEAN achieved this goal by introducing another software layer, the receiver agent. \par

The receiver agent was implemented by following a similar framework to that presented by \citet{schroeder2012towards}. This agent is responsible for reordering goods and receiving deliveries from carriers. The receiver agent defines the order quantity (delivery size) and the delivery time window. For each of these parameters the receiver specifies certain criteria that have to be met by the carrier. These imposed requirements are referred to as the reordering requirements. The receiver behaviour was integrated into the existing framework of the carrier agents in \acrshort{matsim}. \par

ADD UPDATED MATSIM CYCLE

We refer once again to the \acrshort{matsim} cycle. Figure XX shows the updated \acrshort{matsim} cycle incorporating the carrier and receiver interaction. This figure was adapted from BEAN.
During the \textit{initial demand} phase, a group of receiver agents are created. Each of these agents is assigned a facility location and added to the receiver data container. For each group of receivers, a list of product types are generated containing the product description and the capacity required to transport one unit of the product. The relevant product types are then allocated to the receivers. Each product type allocated to an individual receiver is also assigned an initial stock value. Based on these figures and the reordering policy, the daily reorder quantity is calculated and a new receiver order is generated. Each product type is replenished based on the assigned reordering policy. The current policy implemented in \acrshort{matsim} ( by BEAN) is based on the \((s,S)\) reordering policy. The \((s,S)\) reordering policy determines the difference between the maximum inventory level, represented by \(S\), and the level of inventory currently available, below the minimum represented by \(s\). The output from the reordering policy, the weekly order quantity, is converted to a daily order quantity figure for the purpose of the single day simulation executed in \acrshort{matsim}. The proposed structure of the reordering policy allows for the incorporation of alternative reordering policies as it is extendable. Each order is assigned a unique order name, the required unloading time and the preferred number of weekly deliveries.\par

As \acrshort{matsim} simulates a single day, the number of weekly orders required will affect the volume to be delivered per day. A receiver requesting less than five deliveries a week, might not receive a delivery on the specific day simulated (if a five working day policy is assumed). To avoid the instance where a receiver receives zero cost allocation because of no delivery, the weekly delivery cost is estimated. The fixed costs, vehicle capacities, the carrier's cost of time and the number of weekly deliveries requested by the receiver is used to estimate the weekly delivery cost for each receiver.\par

Each generated order is then allocated to a specific carrier. A list of orders to satisfy is compiled for each carrier. The final initial plan allocated to each carrier agent is generated. In the instance where a single receiver requires deliveries from multiple carriers, it is included in this receiver plan. Each order is accompanied by the receiver specified delivery time windows. To integrate the generated receiver order into the existing carrier framework, the order is included as part of the carrier shipment list where the different vehicle types, pickup and delivery locations of the individual carriers are stored. As shown in figure XX, the receiver order is therefore added to the data injected into the \texttt{jsprit} toolkit prior to running the \textit{mobsim} stage. This list contains the unique order name, the receiver facility location, the preferred delivery time windows the size of the order (given in weight or volume) and the unloading duration. The receiver order (injected into the carrier shipment container) is rerouted and scheduled in \texttt{jsprit}. From here, the carrier plan is generated for each of the carrier agents and set as the carriers' initial plan. These plans are then executed in the mobsim stage in the same manner as that discussed for the carrier agent in the previous section. \par

Once the mobsim stage is completed, the process enters the next step in the \acrshort{matsim} cycle, the \textit{scoring} stage. Plans executed in the mobsim are scored using the econometric utility function. The calculated econometric function includes the delivery cost for each of the carriers. To translate these costs to the different receiver agents, the delivery costs are allocated proportionally to the different receivers served by the carrier. The proportional allocation of costs is the basic approached used, but the \acrshort{matsim} infrastructure can accommodate alternative methods. The allocation of costs are discussed in further detail in the following sections. The calculated cost allocated to each of the receivers from the carrier(s) can be seen as the score allocated to the specific plan executed by the different receivers and thus describes the performance of the plan. \par

After the performance of the executed plan has been scored, the cycle enters the \textit{replanning} stage. After each iteration the carrier agent attempts to improve its own situation by altering its delivery plan to improve its performance. Similarly, each receiver agent will also attempt to improve its selected plan's score by reducing the total delivery cost allocated to him. The receiver replanning is done at the beginning of each iteration. This is done by introducing a replanning iteration variable into the framework. This variable is customisable by the modeller and describes after how many intervals the receiver agents are allowed to change their selected plans. If the simulation reaches the specified replanning iteration, the receiver agent has to decide whether its selected plan is sufficient or if any alterations are required. The receiver agent can change its specified plan by applying one of the different replanning strategies. This includes:
\begin{enumerate}
    \item Select the best plan from past plans based on the allocated score
    \item Adjust the delivery time window specified for the product type
    \item Adjust the delivery unloading time
    \item Adjust the delivery frequency 
\end{enumerate}

The new receiver order is injected into \texttt{jsprit} and a new carrier shipment is generated. The new carrier shipment includes the updated order quantity, delivery time windows and delivery unloading times. In \texttt{jsprit} updated plans are rerouted and a new schedule, the carrier plan, is injected into the mobsim stage. From here the cycle is completed and new scored are determined for these executed plans. The score allocated to the specific plan is stored in the agent's memory for future iterations. The completed receiver functionality in \acrshort{matsim} introduced by BEAN, is shown in figure XX. The figure was adapted from BEAN. The completed \acrshort{matsim} cycle includes the receiver and carrier behaviour.

%-------------------------------------------------------------------------------------------------------------------
\section{Impact of receiver reordering parameters on carrier behaviour}
The inclusion of the carrier and receiver agents allows modellers to capture the dynamic interaction between these agents. This allows modellers the capability to dynamically include factors such as congestion due to other freight vehicles and receiver order requirements into the model. These decisions are influenced by the interacting agents and the environment, resulting in more realistic simulation scenarios. Understanding the reason for certain decisions made by agents operating in a certain area, can lead to a better understanding of the freight movement itself to provide more accurate predictions of freight movements in urban areas. Whilst accurate predictions result in more reliable models, it also improves the decisions made by urban freight planners to better prepare for future scenarios.\par

A major contribution of the new software layers is the ability to better analyse and understand the interactions between the carrier and receiver agents. As sustainability has become an important factor to consider it is important that the focus is shifted from the shipper-carrier interaction to that of the receiver and carrier. The receiver has the ability to impact the delivery requirements and is therefore an important player in reducing and improving the utilisation of freight vehicles on the road. A proposed method of improving urban freight movements in urban areas is by implementing collaboration initiatives between receiver and carrier agents. Therefore, the inclusion of the receiver and carrier agents provides a platform for simulating collaboration scenarios.

 In the study conducted by \citet{bean2019behavioural} the carrier response to various receiver reordering parameters was evaluated. The three key receiver reordering behaviours included in this study was the delivery time window, delivery frequency and delivery unloading time introduced by the receiver. For each of the simulation instances the resulting fleet composition, the accompanied fleet utilisation and the utility function was evaluated. Based on the study results, the expected carrier behaviour in this simulation can be predicted by utilising the method and results from \citet{bean2019behavioural}'s study.

%---------------------------------------------------------------------------------------------------
\subsection{Delivery time window}
The delivery time window describes the time interval in which the specific receiver would like to receive the delivery of the product. This interval, referred to as the time window, consists of a start and end time during which the carrier must unload the product at the receiver location. The duration, referred to as the width of the time frame, is specified by the receiver. It is assumed that the receiver is not willing to accept a delivery outside of the specified time frame. If the freight vehicle arrives before the time window start time, the vehicle has to wait at the facility until the start of the acceptable interval. Freight vehicles that missed the allocated delivery time window is considered a late delivery and will receive a penalty cost. Therefore, a delay cost of R6/min is charged in these instances.\par

In the study conducted by \citet{bean2019behavioural}, the delivery time window is randomly scattered throughout the working day, ranging between 06:00 and 18:00. The specified time frame width ranged between 2 and 12 hours, with an increment of 2 hours. The results indicate that different delivery time windows does have an impact on the utilisation of heavy and light vehicles. The results indicate that carriers tend to make use of light vehicles when the delivery time window widths imposed by the receivers are narrower. However, light vehicles tend to be well utilised irrespective of the delivery time window duration imposed.  Heavy vehicles are more likely to be used when the duration is wider as it allows carriers to service multiple receivers in a single delivery trip, while still staying within the boundary of the delivery time windows. Therefore, wider delivery time windows allow more flexibility for planning the deliveries. This flexibility allows the carrier to reduce the accumulated delivery cost by utilising a single heavy vehicle to service the multiple receivers, whilst minimizing the accumulation of late deliveries and the resulting in additional penalty costs. This statement is confirmed by the decrease in the the total carrier delivery cost when the delivery time window duration is increased. 

%-----------------------------------------------------------------------------------------------------
\subsection{Delivery frequency}
Receiver demand is estimated for a specific planning horizon. During this planning horizon, the demand stays constant. The quantity of goods to be delivered in a single delivery trip, depends on the delivery frequency specified by the receiver. A higher delivery frequency will result in more frequent small deliveries, whereas a smaller delivery frequency will result in larger less frequent deliveries. \par

\citet{bean2019behavioural} investigates the impact of different delivery frequencies on the carrier behaviour.  The per-delivery quantity and the probability of a delivery is estimated using Eq. XX and XX, respectively. The weekly demand for each of the receivers, Dr, is divided by the defined number of deliveries defined by each of the receivers, Nr, as shown in Eq. XX. It is important to note that the model implemented by \citet{bean2019behavioural}does not allow for split deliveries where an order is split into multiple deliveries.

Furthermore, the probability of making a delivery, Pr, at the receiver r is calculated as the prescribed number of deliveries per week, Nr, divided by the potential days of making a delivery. It is assumed that only weekly deliveries are allowed resulting in five potential delivery days as indicated by Eq. XX.

The probability of a delivery, Pr, is compared to a random number between 0 and 1, Rr. In the instance where Eq XX is true, (Rr < Pr) a quantity of Q is delivered at receiver R. Else, the condition is false and the specific receiver ,r, does not receive a delivery.\par

Each of the receivers has a specified weekly demand that must be satisfied by the carrier. For the purpose of this scenario, all receivers have a weekly demand of 5 tonne and require a daily delivery. Therefore, the carrier must deliver 1 tonne to each of the receivers.

From the results found by \citet{bean2019behavioural}, it is expected that the carriers tend to make use of heavy vehicles when smaller, more frequent deliveries are required. In this way, the carrier can utilise a single heavy vehicle to service multiple receivers whilst utilising the available capacity of the vehicle. The utilisation of the heavy vehicle capacity is improved when smaller more frequent deliveries are serviced. On the contrary, the carrier will select light vehicles to service receivers that require larger less frequent deliveries, with the exception of a single delivery per week. It can be assumed that this is due to the weekly demand (5 tonne in this scenario) exceeding the total available capacity of the light vehicle (maximum capacity of 3 tonnes). The utilisation of the selected light vehicles also improve when the smaller, more frequent deliveries are required. \par

The resulting delivery cost of the carrier increases when more frequent deliveries of smaller quantities are requested. This is due to the increased use of the light vehicles resulting in an increase in the delivery cost.

%---------------------------------------------------------------------------------------------------
\subsection{Delivery unloading time}
The unloading time specified at each of the receiver locations takes into account the variety of unloading scenarios. Factors such as the delivery size and the unloading conditions can result in different unloading times at the different receiver locations. Factors such as an unloading ramps and the need for material handling equipment such as forklifts and tail-lift have an effect on the unloading time. Furthermore, the unit type such as pallets or boxes also result in different unloading times. \par

\citet{bean2019behavioural} investigates the impact of the delivery unloading time on the behaviour of the carrier. This includes decisions such as the number of vehicles used, vehicle capacity and associated utilisation and the total delivery cost accumulated. The results indicate that carriers tend to make use of heavy vehicles with smaller unloading times. Whereas the lighter vehicles are used when the unloading times are longer in an attempt to minimise the penalty costs for late deliveries. This is evident in the resulting delivery costs incurred. With increased unloading times, the total delivery cost increases as the penalty cost for late deliveries increases.

%------------------------------------------------------------------------------------------------------------------------------------------------------------------------------------------------
\section{Conclusion}
This chapter focused on the investigating the available framework in \acrshort{matsim} as well as recent contributions modelling freight agents. Two new software layers were introduced in \acrshort{matsim} allowing the modelling of logistic behaviour between freight agents. The first contribution was the carrier agent. This extension produces a delivery schedule accompanied by the planned route through the network.  The second contribution was the receiver agent. This powerful agent was embedded in a similar manner than the carrier agent. The importance of this agent lies in its ability to impact the delivery requirements by imposing certain reordering requirements. The three key reordering decisions available to the receiver all have a significant impact on the vehicle composition, utilisation and total delivery cost incurred by the carrier. The carrier agent will plan deliveries in such a manner than the total delivery cost and utilisation of selected vehicles is minimised. However, the delivery time window, frequency and unloading time constraints can prohibit the carrier from achieving these objectives. By means of collaboration, there exists the opportunity to reduce the total delivery cost and improve the utilisation of vehicle. Therefore, finding a balance between the constraints set by the receiver and the delivery cost incurred by the carrier could be beneficial to multiple parties involved.

%This chapter discusses the modules currently available in \acrshort{matsim}. There are, however, some modules that would have been beneficial to this study that are still under development. This includes the contribution of the third freight agent, the shipper agent. The inclusion of this agent would allow the investigation of the shipper-carrier-receiver interaction where the shipper does not act as the carrier but is rather considered an individual with its own logistic behaviours. Another module with potential contributions is the calculation of the cost allocations during collaboration. The investigation of existing infrastructure and the extension of this module is discussed in the next chapter.
%--------------------------------------------------------------------------------------------------------------------------------------------------------------------------------------------------------------------------------------------------------------------------------------------------------