\label{chapter 5}

\section{Defining Urban Freight Transportation within the context of South Africa}
In order to provide a realistic case study, it is necessary to define urban freight transportation within the context of South Africa. This is done to provide an understanding of the additional costs and negative impacts resulting from unsustainable and inefficient urban freight practices.\par

If the focus is turned towards a specific South African environment, the additional costs emerging from the unsustainable logistic practices are still evident . \citet{markman2003gauteng} describes the traffic in South Africa, specifically Gauteng, as a major cause for concern. The high levels of congestion on Gauteng roads lead to additional costs, lower productivity figures, increased travel times and increased fuel consumption's.\par


To combat the congestion in urban areas, articles providing different arguments and approaches were explored. The first argument in reducing congestion states that focus should not be placed on improving road infrastructure, as these roads will soon attract new vehicles only resulting in another source of congestion. Rather the focus should be on improving the public transport \citep{garner2001towards}. However, \citet{garner2001towards} concludes that this argument is unrealistic in the context of South Africa. Both public transportation and road infrastructure need to be developed thus requiring vigorous research and investigation. A second attempt towards reducing congestion was identified by the Gauteng Department of Public Transport, Roads and Works ("Gautrans"). This pilot program imposed certain restrictions on heavy vehicles travelling on the Gauteng freeways \citep{markman2003gauteng}. In an attempt to regulate the number of urban freight vehicles entering urban areas, the Integrated Master Plan recommended to reduce the number of freight vehicles entering city's border, with the aim to reduce the costs accumulated by congestion and crashes. This is similar to restrictions introduced by European cities in the form of time, vehicle weight, size and route restrictions. The enforcement of these restrictions presented many problems as the co-operation of freight transport drivers were difficult to obtain. The introduction of law enforcement in the form of traffic fines and traffic officers were ignored by these parties. \par
%How have these restrictions improved the management of freight vehicles and did it address the negative impacts such as congestion and deterioration of infrastructure?
Although authorities have introduced policies to manage the freight vehicles in urban areas, congestion and delays remain a problematic area for decision makers. Therefore, there is a need for additional planning procedures that can further improve the management of freight vehicles in Gauteng.
This is echoed by \citet{heyns2013traffic} that stresses the need for a traffic congestion management plan to be included in the South African planning systems. This, however, requires a coordinated approach from policy makers and such urban-planning strategies are the responsibility of local municipalities \citep{heyns2013traffic}.\par
% Find references for the following statements
Researchers have attempted to improve the efficiency of this process to obtain promising solutions faster. This is a key component in improving the planning policies of Gauteng roads. The outcome of these planning processes cannot be controlled by other agents than the administrator. Other agents in the supply chain (receivers, carriers and shippers) cannot contribute to reducing congestion in this manner, even though these agents absorb a portion of the costs resulting from congestion. \par

Urban transportation planning strategies in South Africa has not yet reached a stable state. Insufficient infrastructure maintenance, limited public transport and obstruction of enforced laws such as imposed restrictions, remains a problematic factor for supply chain actors.\\


